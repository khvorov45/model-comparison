The scaled logit model is the same as logistic regression except that it estimates the baseline probability of outcome (i.e., the probability at low covariate values assuming a protective covariate) as opposed to assuming that it is equal to 1 \cite{Dunning;2006}. If there is only one covariate which is the antibody titre measurement then the model is

\begin{align*}
    \begin{gathered}
        P(Y=1) = \frac{\lambda}{1 + \text{exp}(\beta_0 + \beta_X X)}
    \end{gathered}
\end{align*}

Note that with the above parameterisation, the $\beta_0$ and $\beta_X$ parameters are negated relative to logistic regression. The new parameter $\lambda$ is the baseline risk, $\text{exp}(\beta_0)$ is the reduction in the baseline risk (protection) at $X=0$, $\beta_X$ is the ratio of protection of subjects with a given $X$ compared to subjects whose $X$ is 1 unit lower.

The baseline risk parameter $\lambda$ is bounded to $[0, 1]$. This may present problems for optimisation. The model can be parameterised as follows to avoid bounded parameters

\begin{gather*}
    P(Y=1) = \frac{\text{exp}(\theta)}{(1 + \text{exp}(\theta))(1 + \text{exp}(\beta_0 + \beta_X X))}
\end{gather*}

Where $\theta$ is baseline log-odds rather than baseline probability.

$$
    \theta = \text{log}\frac{\lambda}{1 - \lambda}
$$

With either parameterisation, the protection is

\begin{gather}
    D(X) = 1 - \frac{\hat{P}(Y=1)}{\hat{\lambda}} = \frac{\text{exp}(\hat{\beta}_0 + \hat{\beta}_X X)}{1 + \text{exp}(\hat{\beta}_0 + \hat{\beta}_X X)}
    \label{eq:sclr-prot}
\end{gather}

The quantity in Eq.~\ref{eq:sclr-prot} represents the reduction in the baseline probablity of infection at a given $X$.
