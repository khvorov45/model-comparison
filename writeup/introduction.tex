In the study of infectious diseases it is useful to have some measure of whether a person is infected, immune or susceptible to infection. This may be important, for example, to understand disease prevalence, population-level susceptibility or for evaluation of vaccines.  However, direct measurement of immunity is often not possible and instead some \textit{correlate} of protection is used. For viral infectious diseases, an oft-used correlate is the serum antibody titre, which provides a measure of the amount of antibody that recognizes a particular epitope. 

Antibody titres have several limitations. Titres are the inverse of the greatest dilution of antibody that inhibits virus in serial dilutions, with higher values indicating greater inhibition. There is no true zero value, nor is there a true measure of the maximum value; titres can only be as low as the minimum starting concentration and as high as the maximum dilution.  Furthermore, within dilution intervals, the true concentration that inhibits virus is unknown; only the upper and lower bound of each dilution is known. In addition, antibody titres are merely a surrogate measure and their sensitivity and specificity may be imperfect, such that reduced titres may not always correspond to increased susceptibility. 

For influenza vaccines, the haemagglutination inhibition (HI) antibody titre is an established correlate of protection. Indeed, the annual reformulation of influenza vaccines is partially dependent on demonstration that circulating viruses are no longer inhibited by vaccine-induced antibodies indicated by HI titres below a certain threshold \citep{Barr;2014}. And annual re-licensing of updated formulations is dependent on demonstrating that a vaccine induces HI titres above this same threshold \citep{Wood;2003}.  The HI threshold commonly used is a titre of 1:40, thought correspond to a 50\% reduction in risk. This figure is derived from cohort studies among vaccinated or infected individuals who have been followed for infection, and among whom the median titre associated with protection (no detected infection) is calculated \citep{Hobson;1972}\citep{Ng;2013}. 

%This threshold is also used to interpret susceptibility to infection among unvaccinated individuals for modelling studies and public health interventions. For example during the 2009 pandemic, age-specific susceptibility, established by sero-epidemiology studies \citep{Hardelid;2011}, helped direct targeted vaccination programmes %[need ref]. 

Several methods have been proposed for the analysis of antibody titre data and the calculation of protective thresholds. Each has its own set of assumptions that makes it more or less appropriate for the data being analysed. Here we will consider 3 models that we have seen used in the literature: Cox proportional hazards regression; ordinary logistic regression; and a scaled logit model. Using simulations and data from two published studies, we discuss these models' assumptions, limitations and situations in which they may not be appropriate.
