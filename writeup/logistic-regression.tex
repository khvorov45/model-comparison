The logistic model assumes that the probability of outcome follows a logistic curve from 1 (at low covariate values assuming a protective covariate) to 0 (at high covariate values). If there is only one covariate which is the antibody titre measurement, then the model is:

\begin{align*}
    \begin{gathered}
        P(Y=1) = \frac{\text{exp}(\beta_0 + \beta_X X)}{1 + \text{exp}(\beta_0 + \beta_X X)} = \text{logit}^{-1}(\beta_0 + \beta_X X)
    \end{gathered}
\end{align*}

Where $\beta_0$ is the log-odds of infection when $X=0$ and $\beta_X$ is the log-odds ratio of infection of subjects with a given $X$ compared to subjects subjects whose $X$ is 1 unit lower. Note that the linear predictor $L(X)$ is equal to $\beta_0 + \beta_X X$.

Once the model is fitted, estimates of $\beta_0$ ($\hat{\beta}_0$), $\beta_X$ ($\hat{\beta}_X$) are obtained, as well as their variance-covariance matrix which has three terms --- $\text{Var}(\hat{\beta}_0)$, $\text{Var}(\hat{\beta}_X)$ and $\text{Cov}(\hat{\beta}_0, \hat{\beta}_X)$.

One way to generate a protection curve from this model is to subtract the fitted probability $\hat{P}$ from 1

\begin{gather}
    D(X) = 1 - \hat{P}(Y=1)
    \label{eq:lr-prot-abs}
\end{gather}

The quantity in Eq.~\ref{eq:lr-prot-abs} represents the probability of not getting infected (i.e., the probability of being protected) at a given $X$.

The confidence interval for the quantity in Eq.~\ref{eq:lr-prot-abs} can be generated from the interval for $L(X)$ as follows

\begin{gather*}
    \big(\hat{D}(X)_{\text{low}},~\hat{D}(X)_{\text{high}}\big)  =
    \big(
    1 - \text{logit}^{-1}(\hat{L}(X)_{\text{high}}),~
    1 - \text{logit}^{-1}(\hat{L}(X)_{\text{low}})
    \big)                                                       \\
    \big(\hat{L}(X)_{\text{low}},~\hat{L}(X)_{\text{high}}\big)  =
    \big(
    \hat{L}(X) - 1.96 \sqrt{\text{Var}(\hat{L}(X))},~
    \hat{L}(X) + 1.96 \sqrt{\text{Var}(\hat{L}(X))}
    \big)                                                       \\
    \hat{L}(X) = \hat{\beta}_0 + \hat{\beta}_XX \quad
    \text{Var}(\hat{L}(X)) = X^2\text{Var}(\hat{\beta}_X) + \text{Var}(\hat{\beta}_0) + X \text{Cov}(\beta_0, \beta_X)
\end{gather*}

Another way to generate a protection curve from the logistic model is to divide the fitted probability of infection at a given $X$ by the fitted propbability of infection when $X$ is equal to some threshold, e.g. $\text{log}(5)$, to obtain the relative probability of infection and then subtract this quantity from 1

\begin{gather}
    D(X) = 1 - \frac{\hat{P}(Y=1 | X)}{\hat{P}(Y=1 | X = \text{log}(5))}
    \label{eq:lr-prot-rel}
\end{gather}

The quantity in Eq.~\ref{eq:lr-prot-rel} represents the probability of being protected at a given $X$ relative to the probability of protection when $X$ is equal to $\text{log}(5)$.

Note that it is difficult to obtain the confidence interval of the quantity in Eq.~\ref{eq:lr-prot-rel} analytically, so a method such as bootstrapping may be required.
Bootstrapping involves generating random resamples from the data, fitting the model to each one and generating the quantity in Eq.~\ref{eq:lr-prot-rel} at a range of $X$ values for each obtained estimate of $\beta_0$ and $\beta_X$.
After fitting the model to $n$ resamples of the original data, $n$ estimates of protection would be obtained at a range of $X$ values.
The 2.5\% and 97.5\% quantiles of the distribution of protection at any $X$ value formed by the $n$ estimates of protection obtained by bootstrapping can serve as the bounds of the 95\% confidence interval for protection at that $X$ value.
